% !TeX program = lualatex
\documentclass[10pt,a4paper]{article}
\usepackage[utf8]{inputenc}
\usepackage[compat=1.1.0]{tikz-feynman}

\newcommand{\diaProlog}{}
\newcommand{\diaEpilog}{\\ \\}


\tikzfeynmanset{
	qgElName/.style={particle=\(e^-\)},
	qgAelName/.style={particle=\(e^+\)},
	qgGaName/.style={particle=\(\gamma\)},
	qgElStyle/.style={fermion},
	qgAelStyle/.style={anti fermion},
	qgGaStyle/.style={boson, blue},
	%	qgElMomentumArrowStyle/.style n args={2}{{momentum'={[arrow style=red]{\(#1\)}}}},
	%	qgAelMomentumArrowStyle/.style n args={2}{{momentum'={[arrow style=red]{\(#1\)}}}},
	%    qgGaMomentumArrowStyle/.style n args={2}{{momentum'={[arrow style=red]{\(#1\)}}}}
	qgElMomentumArrowStyle/.style n args={2}{},
	qgAelMomentumArrowStyle/.style n args={2}{},
	qgGaMomentumArrowStyle/.style n args={2}{}
}


\tikzset{
	qgGaEdgeName/.style={edge label=\(\gamma\)},
	qgElEdgeName/.style={edge label=\(e^-\)},
	qgAelEdgeName/.style={edge label=\(e^+\)},
}

\begin{document}
% File generated by qgraf-3.6.5 via FeynHelpers
% Original input:
% output= 'diagrams.m';
% style= 'tikz-feynman.sty';
% model= '3dFull446';
% in= phi[p];
% out= phi[p];
% loops= 2;
% loop_momentum= k;
% options= onepi, notadpole;



\diaProlog
\feynmandiagram[horizontal=in1 to out1]{
in1 [qgphiName] -- [qgphiStyle, qgphiMomentumArrowStyle={p}{}] v1,
v1 -- [qgphiStyle, qgphiMomentumArrowStyle={p}{}] out1 [qgphiName],
v1 -- [half left, qgphiEdgeName, qgphiStyle, qgphiMomentumArrowStyle={k1}{}] v1,
v1 -- [half left, qgphiEdgeName, qgphiStyle, qgphiMomentumArrowStyle={k2}{}] v1,
};
\diaEpilog
\diaProlog
\feynmandiagram[horizontal=in1 to out1]{
in1 [qgphiName] -- [qgphiStyle, qgphiMomentumArrowStyle={p}{}] v1,
v1 -- [qgphiStyle, qgphiMomentumArrowStyle={p}{}] out1 [qgphiName],
v2 -- [half right, qgphiEdgeName, qgphiStyle, qgphiMomentumArrowStyle={k1}{}] v1,
v2 -- [half left, qgphiEdgeName, qgphiStyle, qgphiMomentumArrowStyle={-k1}{}] v1,
v2 -- [qgphiEdgeName, qgphiStyle, qgphiMomentumArrowStyle={k2}{}] v2,
};
\diaEpilog
\diaProlog
\feynmandiagram[horizontal=in1 to out1]{
in1 [qgphiName] -- [qgphiStyle, qgphiMomentumArrowStyle={p}{}] v1,
v1 -- [qgphiStyle, qgphiMomentumArrowStyle={p}{}] out1 [qgphiName],
v2 -- [half right, qgphiEdgeName, qgphiStyle, qgphiMomentumArrowStyle={k1}{}] v1,
v2 -- [half left, qgphiEdgeName, qgphiStyle, qgphiMomentumArrowStyle={-k1}{}] v1,
v2 -- [qgpsiEdgeName, qgpsiStyle, qgpsiMomentumArrowStyle={k2}{}] v2,
};
\diaEpilog
\diaProlog
\feynmandiagram[horizontal=in1 to out1]{
in1 [qgphiName] -- [qgphiStyle, qgphiMomentumArrowStyle={p}{}] v1,
v1 -- [qgphiStyle, qgphiMomentumArrowStyle={p}{}] out1 [qgphiName],
v2 -- [half left, qgpsiEdgeName, qgpsiStyle, qgpsiMomentumArrowStyle={k1}{}] v1,
v1 -- [half left, qgpsiEdgeName, qgpsiStyle, qgpsiMomentumArrowStyle={k1}{}] v2,
v2 -- [qgphiEdgeName, qgphiStyle, qgphiMomentumArrowStyle={k2}{}] v2,
};
\diaEpilog
\diaProlog
\feynmandiagram[horizontal=in1 to out1]{
in1 [qgphiName] -- [qgphiStyle, qgphiMomentumArrowStyle={p}{}] v1,
v2 -- [qgphiStyle, qgphiMomentumArrowStyle={p}{}] out1 [qgphiName],
v2 -- [half right, qgphiEdgeName, qgphiStyle, qgphiMomentumArrowStyle={k1+k2-p}{}] v1,
v2 -- [half left, qgphiEdgeName, qgphiStyle, qgphiMomentumArrowStyle={-k1}{}] v1,
v2 -- [qgphiEdgeName, qgphiStyle, qgphiMomentumArrowStyle={-k2}{}] v1,
};
\diaEpilog
\diaProlog
\feynmandiagram[horizontal=in1 to out1]{
in1 [qgphiName] -- [qgphiStyle, qgphiMomentumArrowStyle={p}{}] v1,
v2 -- [qgphiStyle, qgphiMomentumArrowStyle={p}{}] out1 [qgphiName],
v2 -- [half right, qgphiEdgeName, qgphiStyle, qgphiMomentumArrowStyle={k1+k2-p}{}] v1,
v2 -- [half left, qgpsiEdgeName, qgpsiStyle, qgpsiMomentumArrowStyle={-k1}{}] v1,
v1 -- [qgpsiEdgeName, qgpsiStyle, qgpsiMomentumArrowStyle={k2}{}] v2,
};
\diaEpilog


\end{document}